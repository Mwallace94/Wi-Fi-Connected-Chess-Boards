\documentclass{article}
\usepackage{geometry}
\usepackage{graphicx}

\usepackage{hyperref}

% Title
\title{Wifi Connected Chess Boards \\ \large Feasibility Report}
\author{Nick Kraus, Kyle Jameson, Maurice Wallace, Mark Mauriello}
\date{\today}

% Margins
\geometry{letterpaper, portrait, margin=.75in}

\begin{document}

\maketitle

%-----------------------
%		Hardware	
%-----------------------

\section*{Hardware Feasibility}
\indent

The hardware that will be necessary for this project has been available for quite a while. Embedded systems are constantly becoming more powerful and cheaper as time goes on. Todays microcontrollers will provide more than enough computing power for running a user application/interface and controlling the mechanical system. For a project like this all the necessary components for the mechanical system have been implemented for other kinds of projects, often to more complex degrees than will be necessary here.

\subsubsection*{Embedded System}
\indent

The embedded system will most likely consist of a Cypress PSoC microcontroller as well as an ESP8266 WiFi module. These will communicate with each other over a UART bus [1]. These kinds of microcontrollers are very well documented, with open source drivers available from Cypress for working with the hardware. The ESP8266 is a very inexpensive Chinese WiFi module that has found a large following in the maker community, leading to a lot of community based libraries, platforms, example projects and much more [2]. Stepper motors have become easy to interface to with simple step/direction motor drivers available from many silicon vendors. These will allow the microcontrollers to control the stepper motors through very simple signaling[3]. Reed switches are also easy to interface to and are very cheap in any quantity. All together many of the electrical systems components will be similar to those made for 3D printers or CNC plotting machines to which there are many well documented projects.

\subsubsection*{Mechanical System}
\indent

The mechanical system will also have many components similar to 3D printers and CNC machines, such as the gantry used to move the pieces and stepper motors used to control the gantry. These components have found common use in industrial machines such as CNC routers and laser cutters [4]. The drive system for the gantry will be a high test fishing line and pulleys, which is a low cost linear motion solution that has found use in delta 3D printer designs [5]. An electromagnet placed on the gantry will be used to physically move the chess pieces, a method of moving objects which has been proven effective by use in commercial chess systems before at a more expensive price point [6]. These components individually have all found previous use in other areas, but will work efficiently together to make the mechanical chess board system.

%-----------------------
%		Software	
%-----------------------

\section*{Software Feasibility}

\subsubsection*{Firmware}
\indent

The firmware for the microcontrollers will be in many ways less complex than that of many CNC machining systems, and yet these systems have been implemented on microcontrollers with far less resources than the proposed processor [7]. In fact, full chess games have been implemented using minuscule resources before, such as a program that takes up a measly 468 bytes including an opponent to play against [8]. The interfacing with stepper motors, lcd displays and reading of reed switches is also a very common application of a microcontroller. While it will need to be well designed to fit within the hardware constraints, there is no reason that a very effective firmware won't be able to complete all the tasks required to create a successful prototype. The ESP8266 will also run its own firmware, allowing it to connect to WiFi networks and the server. With the help of some community projects it has become exceedingly easy to use this device. While the exact technology used to program the ESP8266 is yet to be determined, a good example of its ease of use is a firmware allowing lua code to be run on the ESP8266 [9]. Using this firmware an HTTP client using the TCP protocol can be created in just a few lines of code.

\subsubsection*{Server Software}
\indent

The server software will very possibly be easier to implement than the firmware, due to the hardware being much less constrained. The server will also host as little features as possible, ideally, in order to ease compatibility between many different server systems. This means that the server will only host basic functionality and mostly be used to act as a communication medium between the two WiFi modules. Simple web servers can be quite easy to write up, and for testing purposes can easily be run on a local network computer [10].

\section*{References}

\begin{enumerate}

	\item \href{http://www.cypress.com/file/50321/download}{PSoC Simple UART Communication}
	\item \href{http://rancidbacon.com/files/kiwicon8/ESP8266_WiFi_Module_Quick_Start_Guide_v_1.0.4.pdf}{ESP8266 WiFi Module Quick Start Guide}
	\item \href{http://www.farnell.com/datasheets/1923364.pdf}{Allegro Stepper Motor Driver Datasheet}
	\item \href{https://www.acsmotioncontrol.com/sites/ACS/UserContent/files/downloads/level1/Application%20Stories/HBOT%20Application%20Story.pdf}{H-Bot Gantry Control}
	\item \href{http://richrap.com/?p=182}{3D Delta Printer driven with Spectra Fishing Line}
	\item \href{https://www.youtube.com/watch?v=BobOCMj1Qhs}{Robot Chess Computer with Magnetic Drive}
	\item \href{https://github.com/grbl/grbl}{CNC Controller Software For AtMega 328}
	\item \href{http://www.diplomatie.gouv.fr/en/french-foreign-policy/economic-diplomacy-foreign-trade/facts-about-france/one-figure-one-fact/article/468-the-number-of-bytes-in-the}{468 Byte Chess Program}
	\item \href{http://nodemcu.com/index_en.html}{NodeMCU ESP8266 Firmware}
	\item \href{http://ruslanspivak.com/lsbaws-part1/}{Making a Simple Web Server}

\end{enumerate}

\end{document}